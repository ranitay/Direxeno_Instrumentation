\section{Experimental Setup}
\label{expSetup}

%%%%%%%%%%%%%%%%%%%%%%%%%%%%%%%%%%%%%%%%%%%%%%%%%%%%%%%%%%%%%%%%%%%%%%%%%%%%%%%%%%%%%%%%%%%%%%%%%%%%%%%%%%%%%%%
%Initiator: Ran Itay
%Last modified by: MM Devi, May 26 2017
%Comment: The detector section has been modified. MMdevi's changes are made in blue
%%%%%%%%%%%%%%%%%%%%%%%%%%%%%%%%%%%%%%%%%%%%%%%%%%%%%%%%%%%%%%%%%%%%%%%%%%%%%%%%%%%%%%%%%%%%%%%%%%%%%%%%%%%%%%%

The experimental setup that is described in this section, is designed to measure, the properties of LXe scintillation. 
However it is designed in a modular way, so it can serve different requirements from different future experiments. 
There are three main building blocks consisting the full setup, The purification and circulation system, the cryogenic 
system, and the detector system. Each building block can be replaced without effecting the others. The full assembly 
(figure.~\ref{fig:fulldet}) is held on three separate wracks, one for the DAQ, while the two others which hold the the 
detector and purification system are joined using a 100mm bar with shock absorbers on both sides.   

\subsection{Gas handling system}
\label{subsec:gas}

A typical LXe detector must keep a high level of purity. Careful selection and meticulously cleaning of all parts before mounting, 
is needed, however is not sufficient. The desired level of most detectors of impurity concentration is at the level of 1 ppb $O_2$ 
equivalent~\cite{Aprile:2009dv}. This is crucial to allow ionization electrons drift for several cm. To reach that level in a 
reasonable amount of time (several days instead of months), continuous purification is needed. The gas system, provides this process, 
alongside with all gas handling operations such as filling and recuperation.

During purification mode, xenon is taken from the chamber (in liquid phase)
passes through a heat exchanger\footnote{GEA GBS100M-24 plate heat exchanger} where it is heated and vapored. Then the xenon is forced 
by a KNF diaphragm pump into a hot getter\footnote{MONO-TORR
PS4-MT15-R-2} which cleans the xenon from most impurities. The xenon
also passes through an MKS Mass Flow Controller\footnote{MKS mass flow controller} (MFC). 

After the xenon is purified, it is delivered back to the cryogenic system through the heat exchanger, there the remained xenon gas is 
liquefied before it continuous back to the chamber. A schematic of this system is shown in fig.~\ref{fig:gasSchematic}.


\begin{figure}[t!]
\centerline{\includegraphics[width=1.\linewidth]{GasSchematics.png}}
\caption{Schematics of the purification system. High pressure valves are indicated as valves with arcs. Needle valves are indicated as 
a valve with an arrow.}
\label{fig:gasSchematic}
\end{figure}
\subsection{Cryogenic System}
\label{subsec:cryo}

Remote cooling is generally used in DM experiments due to background radiating from the cooler to the detector. Although in our system 
this is not of great importance there are still several advantages to remote cooling such as: lowering acoustic noise from the cryo-cooler 
and flexibility to design changes. The cryogenic system is connected on one side to the gas system and on the other to the detector chamber, 
any change in the system (e.g, cooler type or model) requires the change of that specific part without changing the detector nor the gas system.

The system is made out of two chambers, the outer vessel (OV) which holds the insulation vacuum, and the inner vessel (IV) that holds the 
xenon. In addition to the vacuum which prevents heat leaks from diffusion and convection, the entire IV is covered by multi layer aluminized 
Myler to prevent heating via radiation A picture of the detector and the CAD design are shown in Fig~\ref{fig:cryo}. 
\begin{figure}
   \centering
    \begin{subfigure}[b]{0.25\textheight}
    \includegraphics[width=\textwidth]{cryogenic1.png}
    \includegraphics[width=\textwidth]{CryoOpen_small.jpg}
    \end{subfigure}
        \caption{(Top) CAD view of the cryogenic system. (Bottom) Pictue of the cryogenic system. \label{fig:cryo}}
\end{figure}


The OV is made of a 10" CF cube, with ports on all six faces (e.g., FT, pumping ports, view ports). The space between the OV and IV is held 
in vacuum for heat insulation (convection and diffusion). This vacuum is shared with the detector one via a 6" CF flexible bellows.

The IV is made of XXX~\,cm height cylinder with 6" CF flanges on the top and bottom parts of it, and it holds inside xenon. A XXX~" cold 
finger is welded to the top flange of the IV. The design of the cold finger is similar to the design of~\cite{xe100_instr2012}, the inner 
part of the cold finger is made of long fins, therefore the surface area of it is bigger resulting in a better heat transport. The upper 
part of the cold finger is in thermal contact with the cryo-cooler~\footnote{QDrive 20BB 9p6 A 3 AYNBNCO} via a cooper adapter. The copper 
adapter hold two $100\Omega$ pt resistor which are connected to a PID reader\footnote{cryo-con model 18i Cryogenic Temp Monitor} fot 
temperature measurements. A Cartridge-heater is also inserted to the copper adapter for emergency heating. 

The cryo-cooler provides up to 70 W of cooling power,and is connected via a 4 1/2" flange to the OV top flange, and reaching the IV top flange. 
Common cryo-coolers used for xenon experiments, work in maximal cooling mode permanently. The QDrive, instead, has temperature control allowing 
it vary the cooling power, which enables to set the temperature with fluctuations smaller then $0.1~\mathrm{C^{\circ}}$ on the cooler itself.

On the inner side of the bottom flange of the IV a thin SS funnel is installed collecting all LXe drops from the cold finger, and delivering 
them to the  detector part. This flange is connected to the detector part, via a 3 3/8" flexible bellows. This bellows hosts two small pipes 
connected to the circulation system, and a third pipe coming from the funnel, all three pipes deliver LXe whereas the GXe is filling the bellows.
%%%%%%%%%%%%%%%%%%%%%%%%%%%%%%%%%%%%%%%%%%%%%%%%%%%%%%%%%%%%%%%%%%%%%%%%%%%%%%%%%%%%%%%%%%%%%%%%%%%%%%%%%%%%%%%%%%%%%%%%%%%%%%%%%%%%%%%%%%%%%%%%%%%%%%%%%%%%%
\subsection{The Detector}
\label{subsec:det}
NEED TO ADD DESCRIPTION OF PMT HOLDER AND OF BOTTOM FLANGE HOLDER TO PREVENT TURKS 

\mmd{The detector refers to the chamber and its inner assembly that contains the the liquid Xenon bubble, the photomultiplier detectors 
around it and their accessories. This chamber is placed below 
the cryogenic system.}
\sout{The Detector part refers to the whole apparatus below the cryogenic system.} 
\mmd{We describe the detector chamber and its interface to the cryogenic system in section \ref{subsubsec:detchamber}. In the two 
subsequent sections we discuss the optical assembly that holds the liquid Xenon and the Photomultiplier detectors.}

%%%%%%%%%%%%%%%%%%%%%%%%%%%%%%%%%%%%%%%%%%%%%%%%%%%%%%%%%%%%%%%%%%%%%%%%%%%%%%%%%%%%%%%%%%%%%%%%%%%%%%%%
\subsubsection{\mmd{The detector chamber and its interface to the cryogenic system}}
\label{subsubsec:detchamber}

\mmd{The chamber} is built such that apart from the interface to 
the cryogenic system, it can be changed and modified easily for future experiments.
The interface unit is built out of 2 flanges welded together via 7 tubes, which serve as service ports for electrical and other feedthroughs. 
The upper flange, ISO-K 160, is part of the outer vessel and shares the insulation vacuum of the cryogenic system, the inner one , CF-8", is 
part of an inner vessel for future detectors, and would hold xenon inside. For our experiment we modified the CF flange to fit also a XXX" CF 
flange which we use.

The OV is closed with a cylinder XXX~\,cm height closed from the bottom with another ISO-K 160 flange, the height of the cylinder is determined 
such that the maximal height of the whole apparatus is 190~\,cm, allowing the mobility of the detector through standard doors.
 
The XXX" CF flange is connected to a closed vessel internally divided into two parts. This vessel serves as a xenon reservoir. The two parts of 
the vessel are connected to a spherical orb from above (inner part) and from below (outer part). LXe is circulated such that new LXe drips into 
the outer part and pumped from the inner one. This way the liquid level is controlled, and the sphere itself will always be filled with LXe. 

The main part of the detector is the spherical orb, which is made of fused silica. In the center of it, a smaller sphere is curved to hold the 
LXe, two invar pipes are connected to it from the top and bottom with SS mini-CF flanges at the end, to circulate the xenon (see Fig.~\ref{fig:sphere}). 
The sphere stands in the center of 20 PMTs\footnote{r8520-406 Hamamatsu 1" PMT} to detect light emitted from the LXe.

The bottom flange of the sphere is held using a brass holder to prevent force or torque applied on the sphere while mounting the detector. The 
brass holder is connected to a plate held from the top 8" flange, and is also used to allign this plate at first installation. 

%%%%%%%%%%%%%%%%%%%%%%%%%%%%%%%%%%%%%%%%%%%%%%%%%%%%%%%%%%%%%%%%%%%%%%%%%%%%%%%%%%%%%%%%%%%%%%%%%%%%%%%%%%%%%%%%%%%%%%%
\subsubsection{\mmd{The HPFS Shell to contain LXe}}
\label{subsubsec:detchamber}

The LXe target bubble should not be too large in order to avoid double scatters. A spherical shell made of high purity fused silica 
with high transmittance is designed to hold the LXe target. This shell should be large enough to reduce internal reflections, but not too large which would 
attenuate the scintillation light. Also the material of the shell should have a refractive index similar to LXe in order to have 
minimal change in the direction of the photons coming from the target. We chose corning HPFS 8655 as the shell material. According 
to the factsheet, the refractive index of HPFS 8655 is 1.575 at 185 nm (LXe R.I. 1.61) and the transmittance 99.8\%/cm at 175 nm. 
In Fig.~\ref{fig:hpfsRIcalibration}, we show the refractive indices at various wavelegths as given by the HPFS factsheet and also 
a naive extrapolation at lower wavelengths which are relevant to us.

\begin{figure}
   \centering
   \includegraphics[width=0.8\textwidth]{RI-calibration.pdf}
   \caption{The refaractive indices of HPFS 8655 at various wavelengths. (Top) The values provided from Corning factsheet.
   (Bottom) The values extrapolated at lower wavelengths.} 
   \label{fig:hpfsRIcalibration}
\end{figure}

The transmittance of the material is extremely crucial for us to optimize the dimension of the fused silica shell. For that 
reason, we obtained a 6 mm thick sample of HPFS 8655 and performed a transmittance testusing a VUV monochromator setup. 
A deuterium light source was used to generate a spectrum in
the range 110 -- 950 nm, peaked approximately at 160 nm. The window of the light source faced a vacuum
space pumped to below $10^{-4}$ Torr. The monochromator allows to select
the desired wavelengths using a manually rotatable holographic diffraction grating. A PMT placed
in the vacuum measured the intensity of light emitted from the monochromator, with and without
the fused silica sample. The ratio of measured intensities was used to calculate the transmittance
of the material. Fig.~\ref{fig:transmittance} shows the measured transmittances/ 6 mm at 
150 nm -- 215 nm. At 175 nm the sample shows approximately 90\% measured transmittance/ 6mm which corresponds 
to an intrisnic transmittance of about 98\%.  

\begin{figure}
   \centering
   \includegraphics[width=0.6\textwidth]{ObservedTransmittance1.pdf}
   \caption{The transmittance of a 6 mm thick HPFS 8655 sample.} 
   \label{fig:transmittance}
\end{figure}

The dimension of the fused silica shell is optimized by studying the path of the scintillation photons 
using a GEANT4 based simulation. 
The sources that will be used for exciting the xenon, and creating the supperradiance (signal) as well as the standard emission (background), 
will be $^{137} \mathrm{Cs}$ (662 keV) and $^{57} \mathrm{Co}$(122keV \& 136 keV) for ER and $^241$AmBe , D-D neutron generator, or neutron 
produced in an accelerator for NR . The mean free path for this energy is a couple of mm ($^{57} \mathrm{Co}$) and 0.5-3 cm ($^{137} Cs$).  
We discuss the simulation in detail in section~\ref{sec:simulation}. 
The outer radius of the shell is 3 cm, while the inner radius of the hollow space that will hold the 
LXe is 1 cm. The flow of the LXe will be maintained by two invar tubes. This shell--system, as shown 
in Fig.~\ref{fig:sphere} is being manufactured industrially as par the specification provided by us.

\begin{figure}
   \centering
   \includegraphics[width=0.6\textwidth]{sphere.png}
   \caption{The technical scheme of the HPFS shell with invar tubing and flanges.} 
   \label{fig:sphere}
\end{figure}


The photons coming out of the system will be detected by twenty 1'' square Hamamatsu R8520-406 photomultiplier 
tubes with an active area of 20.5 mm $\times$ 20.5 mm each. We pick PMTs with a minimum quantum efficiency of 30\% 
at 178 nm. For an applied voltage of 900 V the gain of these PMTs are ~ 2 $\times$ 10$^6$. We use a positive 
voltage divider, also manufactured by Hamamatsu, to provide high voltage to the PMTs.
These 20 PMTs are held with a special aluminum holder, coated with anti-reflection substance. The holder is made of two hemispheres hosting the PMTs in 
3 rows all of them pointing to the center of the fused-silica sphere. the PMTs are held only via their voltage--divider bases. The bases are held using M2 PEEK screws. 
In Fig.~\ref{fig:pmtholder}, one of the holder--hemispheres with the PMTs are shown.

\begin{figure}
   \centering
   \includegraphics[width=0.6\textwidth]{IMG_7067.JPG}
   \caption{A PMT holder--hemisphere. We use two such components to hold 20 PMTS around the target.} 
   \label{fig:pmtholder}
\end{figure}


%\clearpage %temporary MMD
In Fig.~\ref{fig:detector} we present a CAD schematic as well as a real view of the detector part.

%%%%%%%%%%%%%%%%%%%%%%%%%%%%%%%%%%%%%%%%%%%%%%%%%%%%%%%%%%%%%%%%%%%%%%%%%%
\begin{figure}
	\centering
    \begin{subfigure}[b]{0.45\textwidth}
		\includegraphics[width=0.75\textwidth , height=0.3\textheight]{detCAD.png}% Here is how to import 
	\end{subfigure}	
	\begin{subfigure}[b]{0.45\textwidth}
		\includegraphics[width=\textwidth , height=0.3\textheight]{detReal_small.jpg}% Here is how to import 
	\end{subfigure}	
		\caption{\label{fig:detector} (Left) CAD design of the detector part. (Right) First mounting of the 
		detector part, still not connected to the rest of the system.}
	
\end{figure}
%%%%%%%%%%%%%%%%%%%%%%%%%%%%%%%%%%%%%%%%%%%%%%%%%%%%%%%%%%%%%%%%%%%%%%%%%%%%
