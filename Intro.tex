\section{Introduction}
\label{sec:Intro}

The use of Noble-liquid detectors in the field of astroparticle physics has increased in the past decades. 
Detectors aiming at measuring Dark Matter (DM) particles and neutrinos properties use large liquid argon or 
liquid xenon (LXe) chambers~\cite{Aprile:2009dv,Rubbia:2013tpa}. Current and future experiment for DM detectioin 
are tuned to detect weakly interacting massive particles (WIMPs), a postulated candiadte for DM particle~\cite{Bertone:2010zza}. 
LXe based detectors are to date the leading in sensitivity and size for these searches~\cite{Aprile:2017iyp,Akerib:2016vxi,Fu:2016ega,Aalbers:2016jon}. 

When a particle interacts within the LXe media, it forms a cloud of excited and ionized states with typical length of 100~nm. 
The excited Xe ($Xe^*$) combines with other Xe atoms to form an excited dimer state (excimer) when they decay to ground state they emit light. 
\begin{equation} \label{eq:XeSci1}
 Xe^*+Xe \rightarrow Xe^*_2 \rightarrow 2Xe + h \nu , 
\end{equation}
The electrons emitted from the ionization can recombine with a surrounding atom, this process of recombination provides another possibility to produce excimers,
\begin{equation} \label{eq:XeSci2}
\begin{split}
  &Xe^{+} + Xe \rightarrow Xe^{+}_2 \\
  &Xe^{+}_2 + e^{-}  \rightarrow Xe^{**} \\
  &Xe^{**}   \rightarrow Xe^* + heat .\\
  \end{split}
\end{equation}  
Once $Xe^*$ is produced it adds to the scintillation process explained in~\ref{eq:XeSci1}. There are two types of $Xe^*_2$ excimer states, 
singlet and triplet, with lifetime of $\sim3$ ns and $\sim25$ ns respectively. The wavelength emitted by these states is between (175-180)~nm 
which is lower then the lowest excitation of xenon, and therefore travel through it to reach a photo-detector situated outside the LXe. Although 
much is measured on these scintillation processes, the basic knowledge of the quantum properties of these interactions is based on experiments 
preformed several decades ago.

The phenomenon of supperradiance in which identical quantum states ``communicate" through electromagnetic field if in close proximity, 
is well studied. In certain conditions the emission of photons from these correlated states is very different then the sum of random states. 
This difference is in spectral, temporal and spatial properties~\cite{DickeSR,GROSS1982301}. Early studies show that scintillation in LXe can 
produce coherent amplification of light~\cite{BasovSRTheory,MiesSRExp}. These studies were focusing on the macroscopic ionization using high 
energy density electron beams. 

The understanding and quantification of the microscopic effects of non linear phenomena such as supperradiance in Lxe for a single interaction, 
can improve DM experiments to reduce background by the extra knowledge of directionality. irreducible background such as coherent neutrino nucleus 
scattering of neutrinos from the sun can be discard.

In this paper we present an experimental set-up called DIREXENO (DIREctional XENOn) aiming at measuring the spatial distribution of LXe scintillation 
light, and quantify non-isotropic emission.   