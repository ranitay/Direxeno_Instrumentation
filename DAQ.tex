\subsection{The data acquisition system }
\label{sec:DAQ}

%%%%%%%%%%%%%%%%%%%%%%%%%%%%%%%%%%%%%%%%%%%%%%%%%%%%%%%%%%%%%%%%%%%%%%%%%%%%%%%%%%%%
%by MMdevi. Last modified July 2, 2017.
%%%%%%%%%%%%%%%%%%%%%%%%%%%%%%%%%%%%%%%%%%%%%%%%%%%%%%%%%%%%%%%%%%%%%%%%%%%%%%%%%%%%


The DAQ system is heterogeneous using both 
NIM and VME electronic modules. The data readout is being carried out through a PCIe card \footnote{CAEN A3818 PCIe}  connected via an optical link to a VME controller\footnote{CAEN V2718 VME controller}. A schematic layout of the DAQ system is shown in Fig.~{\ref{Fig:DAQscheme}}. 

The PMTs are ramped up to their individual working voltage (corresponding to a gain of $2\times10^6$) using VME high voltage distributor module\footnote{iseg VDS18130p : 
24 independent channels positive polarity voltage distributer}. The raw pulses from the PMTs are amplified and shaped using 
two PMT preamplifiers\footnote{Phillips 776. 16 independent and direct-coupled amplifiers channels}. The preamplifier operates 
from DC to 275 MHz and produces two identical 50 $\Omega$ non inverting outputs with voltage gains of 10 for each PMT channel. One 
of the outputs is converted into a digital signal by an ADC\footnote{CAEN ADC V1742: switched capacitor digitizer}, and the other to 
binary signals using two discriminators\footnote{CAEN V895 16 channel leading edge discriminator}.

%%%%%%%%%%%%%%%%%%%%%%%%%%%%%%%%%%%%%%%%%%%%%%%%%%%%%%%%%%%%%%%%%%%%%%%%%%
\begin{figure}[h]
   \centering
   \includegraphics[width=0.85\textwidth]{DAQscheme.pdf}
   \caption{The schematic of the data acquisition system of \textsc{DireXeno}. The 			signal coming from 20 PMTs ({\it i} = 1 -- 20) and the subsequent electronic channels to record the events once triggered. Where S{\it i}(raw) is the raw electrical pulse output of the PMTs, S{\it i} are the amplified pulses, and SD{\it i} are the binary outputs from the discriminator.
}
   \label{Fig:DAQscheme}
\end{figure}
%%%%%%%%%%%%%%%%%%%%%%%%%%%%%%%%%%%%%%%%%%%%%%%%%%%%%%%%%%%%%%%%%%%%%%%%%%%


The ADC consists of two 12\,bit 5\,GS/s switched capacitor digitizer sections, 
each of them with 16+1 channels, based on DRS4 chip. The dynamic range of the input signal is 1\,Vpp with an adjustable DC offset. This module constantly samples (5\,GS/s, 2.5\,GS/s or 2\,GS/s) either bipolar or unipolar analog input 
signals, and records them into circular 
analog memory buffers. Once triggered, all analog memory 
buffers are frozen and digitized into a digital memory buffer 
with a 12 bit resolution. The measured rise time of the system (PMTs, bases, DAQ) is measured to be ($1.4 \pm 0.6$)\,ns, the measured jitter is, $(390 \pm 67)$\,ps. 

The binary output signals from the discriminator are duplicated and fed to 
the logic module\footnote{CAEN V1495: FPGA based general purpose VME board} and to a scaler\footnote{CAEN V830: 16 channel scalar}. 
A global majority trigger is generated in the logic module with the coincidence of any two out of the twenty PMTs within a predefined time window, that will be optimized to reduce dark counts. The event information and trigger rate are read from the ADC, while the individual PMTs trigger rate from the scaler. Further analyses of the relevant events are carried out offline.

\subsection{\textcolor{blue}{The slow control system}}
\label{subsec:sc}

We use a time-series server based on influxdb (cite: https://www.influxdata.com/). This time series database is built specifically for handling metrics and events or measurements that are time-stamped. 
For monitoring and visualization we use  
Grafana (cite: https://grafana.com), an open source software for time series analytics. 

The monitored data is collected from a variety of sensors and streamed to the database using the influxdb API integrated in python control scripts.
It includes cryogenic system parameters: several temperatures  measured using a PT100 and a cryoconXXXX taken in different points around the sphere 
and in the pools. To gain better understanding of the LXe we increased the number of temperature monitored positions by using a 2-wired 
readout for some sensors, thus allowing the readout of more PT100 sensors with the same number of feedthroughs data channels.
The temperature-resistance calibration curve were shifted accordingly.   
Pressure/Vacuum reading of the outer- and inner-chamber were done using xxxx and xxxx sensors. 
Termistors based sensors were installed on the cooling water lines, the compressor and around the laboratory for monitoring of the cryocooler operation. They were read using an arduino. 

The xenon flow rate was measured using a xxx. A specially designed switch enabeled the transition between Filling/Circulation/Recooperation modes, thus 
allowing the proper handling of the flow rate data in the data base (either adding to the total Xe amount, removing).


The PMTs voltage, current and trigger rate is also monitoted and streamed to the database. 

An optical camera installed. diff ertc.



%\clearpage %temporary TBC
