\section{The Data Acquisition (DAQ) System }
\label{sec:DAQ}

%%%%%%%%%%%%%%%%%%%%%%%%%%%%%%%%%%%%%%%%%%%%%%%%%%%%%%%%%%%%%%%%%%%%%%%%%%%%%%%%%%%%
%by MMdevi. Last modified May 24, 2017.
%%Some of the segments, like trigger logic, noise rate read out might change a bit
%%later, depending on their final implementation.
%%%%%%%%%%%%%%%%%%%%%%%%%%%%%%%%%%%%%%%%%%%%%%%%%%%%%%%%%%%%%%%%%%%%%%%%%%%%%%%%%%%%

In this section we discuss the data acquisition and readout from the PMT 
channels. We use a heterogeneous system consisting of both 
NIM and VME electronics modules, with the data readout being carried out through 
the VME controller module V2718 and a PCIe card A3818. 

The schematic layout 
of the DAQ system is shown in Fig.~{\ref{Fig:DAQscheme}}. All the 20 PMTs 
are oriented in the holder assembly which was discussed in section---. 
The PMTs are ramped up to +800V (the maximum is +900V) using a 24 channel 
VME high volatge distributor module (VDS18130p) from iseg. The voltage is 
monirored by the GUI based control software provided by iseg. 
For clearity in understanding, we term the raw electrical pulse output of 
the PMTs as S{\it i} (raw), {\it i} = 1 -- 20. These raw pulses are then 
amplified and shaped using two NIM photomultiplier preamplifiers (Phillips 776). 
Each of the preamplifier model provides 16 independent  and direct-coupled 
amplifiers channels. The preamplifier channels operates from DC to 275 MHz and 
produce two identical 50 $\Omega$ non inverting outputs with voltage gains of 10. 
We term the amplified pulses as S{\it i}, {\it i} = 1 -- 20. One of the 
two identical analog outputs from each channel is ereingconverted to a digital 
signal with an Analog to Digitizer converter (ADC V1742). 

The ADC module V1742 
is a VME board with two 12bit 5GS/s Switched capacitor Digitizer sections, 
each of them with 16+1 channels, based on DRS4 chip. Th dynamic range of the input 
signal is 1 Vpp with adjustable DC offset. This module can sample either bipolar or 
unipolar analog input signal within the dymnamic range in a circular 
analog memory buffer, with default sampling frequency choices 5GS/s, 2.5 GS/s 
or 2 GS/s. As soon as a trigger signal reaches, all the analog memory 
buffers gets frozen and then gets digitized into a digital memory buffer 
with a 12 bit resolution. 

We generate an intrinsic trigger for the system, 
with the coincidence of any two out of the twenty PMTs. The second output from the 
preamplifier channels are converted to binary signals using two 16 channel leading 
Edge discriminator (V895). In Fig.~\ref{Fig:DAQscheme}, we term the binary 
outputs from V895 as SD{\it i}, i= 1 -- 20. The SD{\it i} signals are then passed over to 
v1495, an FPGA based General purpose VME board which is programmed to perform the 
logic operation to obtain the trigger. At present, the coincidence of any two out 
of the twenty PMTs forms the trigger logic. The output of V1495 logic operation is 
used to trigger the ADC V1742 module. In order to record the PMT noise rate, the 
SD{\it i} signals are duplicated and fed to a scaler V830.


%%%%%%%%%%%%%%%%%%%%%%%%%%%%%%%%%%%%%%%%%%%%%%%%%%%%%%%%%%%%%%%%%%%%%%%%%%
\begin{figure}
   \centering
   \includegraphics[width=\textwidth]{DAQscheme.png}
   \caption{The schematic of the Data Acquisition System of Direxeno. It 
        consists of 20 PMTs and the subsequent electronic channels to record 
        the events for an internal trigger generated by the coincidence of any 
        two PMTs in the system and also the PMT noise rate.}
   \label{Fig:DAQscheme}
\end{figure}
%%%%%%%%%%%%%%%%%%%%%%%%%%%%%%%%%%%%%%%%%%%%%%%%%%%%%%%%%%%%%%%%%%%%%%%%%%%

The PMT event information and the trigger rate are read from the ADC, while the Scaler 
records the PMT noise rate. The data readout to the acquisition PC is done through the 
Controller and optical link to the master PCIe card. The further analyses of the 
relevant events in the PMTs will be carried out offline using an analysis 
framework.





\clearpage %temporary TBC
