\section{Summary}
\label{sec:summary}

The setup of Direxeno, an experiment to measure the spatial and temporal distribution of LXe scintillation, has 
been presented. The system consists of 4 main building blocks (gas handling, cryogenic, detector, 
and DAQ), each of which can be exchanged without altering others allowing significant flexibility 
and modularity. Each of the building blocks has been described in detail, with emphasis on the design 
and components.

The sensitivity of the setup to different postulated non isotropic emission patterns are studied using MC simulations. For the patterns studied, a run--time of 2-3 weeks is required using typical radioactive 
sources. Therefore the system is designed to maintain stability over a reasonable time period.

Using DireXeno, effects like superradiance or any other non-linear scintillation can be measured. 
Measuring the correlation between the direction of the emission and the direction of the radioactive 
source, may lead to directionality measurement which will allow enhanced statistical modeling of 
background and improved sensitivity in DM experiments.
