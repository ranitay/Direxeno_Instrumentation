\section{Summary}
\label{sec:summary}

We have presented Direxeno, a setup to measure the spatial distribution of LXe scintillation. 
The system described is built from 4 main building (gas handling system, cryogenic, detector, 
and DAQ) blocks which can be exchanged without effecting others. This property allows flexibility 
and modularity.

We have presented the sensitivity to different postulated non isotropic emission types, and a 
test statistics to quantify them. For all of them a run time of 2-3 weeks is required assuming 
standard radioactive sources. Therefore the system is designed to maintain stability.

Using Direxeno, effects like superradiance or any other non-linear scintillation can be measured. 
Measuring the correlation between the direction of the emission and the direction of the radioactive 
source, may lead to directionality measurement which will allow reducing of background and improving 
sensitivity in DM experiments.

\vspace{0.5cm}

\mmd{The setup of Direxeno, an experiment to measure the spatial distribution of LXe scintillation, has 
been presented. The system consists of 4 main building blocks (gas handling system, cryogenic, detector, 
and DAQ), each of which can be exchanged without altering others and hence allows significant flexibility 
and modularity. Each of the building blocks has been described in detail, with emphasis on the design 
and components.}

\mmd{The sensitivity of the setup to different postulated non isotropic emission types are studied with a 
GEANT4 Monte Carlo. A test statistics that quantifies each of the emission pattern is performed. 
For all of the patterns studied, a run--time of 2-3 weeks is required assuming standard radioactive 
sources. Therefore the system is designed to maintain stability over a reasonable time period.}

\mmd{Using Direxeno, effects like superradiance or any other non-linear scintillation can be measured. 
Measuring the correlation between the direction of the emission and the direction of the radioactive 
source, may lead to directionality measurement which will allow enhanced statistical modeling of 
background and improved sensitivity in DM experiments.}
