\section{Simulation (\mmd{in progress})}
\label{sec:sim}
%%%%%%%%%%%%%%%%%%%%%%%%%%%%%%%%%%%%%%%%%%%%%%%%%%%%%%%%%%%%%%%%%%%%%%%%%%%%%%%%%%%%%%%%%%%%%%%%%%%%%%%%%%%%%%%%%%%%%%%%
To asses the sensitivity of the detector to identify an emission pattern, a GEANT4 based simulation is used. 
Several emission types and the propagation of photons in the detector are simulated to produce the hit pattern  
at the PMTs for each type. 
Assuming a number of test patterns of the LXe scintillation, the photons detected at the PMTs are mapped and put 
through a statistical test to check the detector's sensitivity towards those patterns.

The realistic dimension of the detector assembly is used in the simulation. The important 
geometrical and optical parameters are listed in Table~\ref{tab:OptPar}. 
The probability for a photon being transmitted/reflected at a given surface is 
determined by Fresnel's equations, which include Snell's law for the transmitted light, 
and specular reflection for the reflected light. The boundary surfaces between different media
such as, the LXe--HPFS, HPFS--vacuum and vacuum--PMT, are assumed to be perfectly smooth, 
therefore enabling only specular reflection.

%%%%%%%%%%%%%%%%%%%%%%%%%%%%%%%%%%%%%%%%%%%%%%%%%%%%%%%%%%%%%%%%%%%%%%%%%%%%%%%%%%%%%%%
\begin{table}[h]
  \centering
  \caption{The parameters used in simulation}
  \label{tab:OptPar}
  \begin{tabular}{|c c||c c|}
  \hline
  Parameter & Value & Parameter & Value \\
  \hline
  LXe absorption length & 100 cm & No. of PMT & 20\\
  %\hline
  LXe scattering length & 35 cm & PMT active area & 22mm $\times$ 22mm\\
  %\hline
  HPFS absorption length & 100 cm & PMT QE & $\geq$ 30\% \\
 % \hline
  HPFS scattering length & $\infty$ & PMT distance from centre & 39 mm\\
 % \hline
  LXe refractive index & 1.61 & LXe bubble radius & 1cm\\
  %\hline
  HPFS refractive index & 1.57 & HPFS shell thickness & 2 cm \\
  Scintillation wavelength & 178 $\pm$ 14 nm & Invar tube diameter & 1 mm\\
  \hline
 \end{tabular}
\end{table}
%%%%%%%%%%%%%%%%%%%%%%%%%%%%%%%%%%%%%%%%%%%%%%%%%%%%%%%%%%%%%%%%%%%%%%%%%%%%%%%%%%%%%%%

The photons reaching the PMTs can either be detected, absorbed or reflected from the photocathode 
or from the PMT window. A simplified approach of the above possibilities is considered -- \RanComment{Not sure nut maybe : instead of --}
A photon \RanComment{reaching/arriving} on the PMT has \RanComment{is considered/assumed to have ?} 30\% probability to be detected (since QE $\geq$ 30\%), 50\% probability to get 
absorbed and 20\% probability to get specularly reflected. \sout{As} the scintillation light \RanComment{from ?}in a particular 
event is emitted by a cloud of \sout{excited diatomic molecules}\RanComment{excimers}. \RanComment{This cloud has/is assumed to have ?} \sout{with} a linear size much smaller than that 
of the optical system, RanComment{. Therefore/Hence,} each event is simulated as a number of photons that are emitted from a point 
in the LXe. The events are uniformly generated in the LXe volume to simulate an entirely radiated 
target much smaller than the mean free path length of the source particles in LXe\RanComment{Not sure I understand what is said here}.

For each scintillation event, a number of photons are detected by the PMTs, with a possibility of 
no photons detected by some of the PMTs\RanComment{Do you think adding due to the above mentioned assumption or something like this?}. The exact position on the photocathode which the photons hit, 
and the exact number of photons falling on a PMT are not known. The electronic signal generated in a PMT 
for a certain number of incident photon is statistical in nature\RanComment{missing the consequence- hence Poisson fluctuations are taken into account?}. The simulation is performed for both 
ideal photon counting and statistical photon counting. The R8520 PMTs also has  ~20\% probability 
for double photoelectric emission for 178 nm photons, which is also included in the simulation.
Each detected photon 
on a PMT is \sout{then} assigned a uniformly random position on the PMT surface. The direction of this point with respect 
to the center of the LXe sphere is defined as the incident direction of the photon. The direction information 
is then used to calculate the angles between all possible pairs of photon \RanComment{for}\sout{in} an\RanComment{y} event and calculate the correlation between all angle pairs.

\RanComment{New Paragraph Start here}

In order to quantify the anisotropicity of the emission, we compare the angle correlation distribution of the anisotropic hit pattern, with the distribution of an isotropic one. we use a $\chi^2$ test statistics where the reduced $\chi^2$ is defined as: 

\begin{equation}
\chi^2/\nu = \frac{1}{\nu} \sum^{\nu}_{i=1} \frac{(O_i - E_i)^2}{E_i},
\label{redchi2}
\end{equation}

where $E_i$ is the pdf of an isotropic emission obtained from a sample of $10^5$ simulated events. $O_i$ is the pdf of the anisotropic emission, and $\nu$ is the total number of angle correlation bins which is also the degree of freedom. In this analysis, 60 bins of identical width are used.

\RanComment{New Paragraph Ends here}

In order to test the emission pattern, the angle correlation distribution of a large sample ($10^5$ events) 
from isotropic emission is considered \RanComment{as}\sout{to be} the PDF of the null hypothesis and is compared 
to that of a number of smaller samples ($10^4$ events) from different types of anisotropic emissions. 
The reduced $\chi^2$ for each sample is calculated as follows.

\begin{equation}
\chi^2/\nu = \frac{1}{\nu} \sum^{\nu}_{i=1} \frac{(O_i - E_i)^2}{E_i}
\label{redchi2}
\end{equation}

where, $O_i$ is the observed count in the $i^{th}$ angle bin, $E_i$ is the expected count in the $i^{th}$ 
angle bin and $\nu$ is the total number of angle bins which is also the degree of freedom. In this analysis, 
60 bins of identical width are used. 


The anisotropic emission patterns used in the analysis are listed 
in Table~\ref{tab:AnisoPattern}. For each emission pattern,  an anisotropic pattern is taken to be embedded 
in an isotropic background. The fraction of photons in the anisotropic pattern ($r_{aniso}$) is 10 \% of the net photons ,
\sout{and} the rest \sout{is} \RanComment{are} the isotropic background. For multiple beams in a pattern $r_{aniso}$ is \sout{splitted} \RanComment{split} among them 
as mentioned in the Table. The intensity of the beams of photons are of Gaussian nature, and their emission direction are 
randomly varied in each event.

%%%%%%%%%%%%%%%%%%%%%%%%%%%%%%%%%%%%%%%%%%%%%%%%%%%%%%%%%%%%%%%%%%%%%%%%%%%%%%%%%%%%%%%
\begin{table}[h]
  \centering
  \caption{Emission patterns. For all patterns $r_{aniso}$ = 0.1 from a 
  total of 50 photon/event.}
  \label{tab:AnisoPattern}
  \begin{tabular}{|c | c| c | c|}
  \hline
  Pattern no. & No. of beams and type & Beam half widths & Signal fractions \\
  %\hline
  1 & 1 & $\sigma_1$ = $5^{0}$ & $r_1$ = 1 \\
  %\hline
   2 & 1 & $\sigma_1$ = $15^{0}$ & $r_1$ = 1 \\
  %\hline
   3 & 2 correlated & $\sigma_1$ = $5^{0}$, $\sigma_2$ = $5^{0}$ & $r_1$ = 0.5, $r_2$ = 0.5  \\
  %\hline
   4 & 2 correlated & $\sigma_1$ = $15^{0}$, $\sigma_2$ = $15^{0}$ & $r_1$ = 0.5, $r_2$ = 0.5 \\
  %\hline
   5 & 2 correlated & $\sigma_1$ = $5^{0}$, $\sigma_2$ = $10^{0}$ & $r_1$ = 0.5, $r_2$ = 0.5 \\
  %\hline
   6 & 2 correlated & $\sigma_1$ = $30^{0}$, $\sigma_2$ = $10^{0}$ & $r_1$ = 0.5, $r_2$ = 0.5 \\
  %\hline
   7 & 2 correlated & $\sigma_1$ = $5^{0}$, $\sigma_2$ = $5^{0}$ & $r_1$ = 0.5, $r_2$ = 0.5 \\
  %\hline
   8 & 2 correlated & $\sigma_1$ = $15^{0}$, $\sigma_2$ = $15^{0}$ & $r_1$ = 0.5, $r_2$ = 0.5 \\
  %\hline
   9 & 2 correlated & $\sigma_1$ = $10^{0}$, $\sigma_2$ = $30^{0}$ & $r_1$ = 0.2, $r_2$ = 0.8 \\
  %\hline
    10 & 2 correlated & $\sigma_1$ = $30^{0}$, $\sigma_2$ = $10^{0}$ & $r_1$ = 0.2, $r_2$ = 0.8 \\
  \hline
 \end{tabular}
\end{table}
%%%%%%%%%%%%%%%%%%%%%%%%%%%%%%%%%%%%%%%%%%%%%%%%%%%%%%%%%%%%%%%%%%%%%%%%%%%%%%%%%%%%%%%

\sout{The number of events needed to detect the anisotropy of a pattern with 
high statistical significance is estimated by calculating $\chi^2/\nu$ 
for a data sample.} 100 data sets generated with different random seeds are 
\sout{used} \RanComment{generated} to obtain the 2$\sigma$ fluctuation on <$\chi^2/\nu$> \RanComment{I dont understand, can this number be calculated, if so lets add the calculation, if its just a number we believe is high enough then no need to add the whole 2 sigma part }. The <$\chi^2/\nu$> 
and its 2$\sigma$ band for pattern 1 \sout{are compared to} \RanComment{overlaid with} the corresponding values 
for an isotropic emission \RanComment{are presented/shown }in Fig.~\ref{fig:pattern1}. The <$\chi^2/\nu$> for 
isotropic emission fluctuate around 1 with $\sigma$ = 0.2 which is consistent with 
the expected value of $\frac{1}{\sqrt{30}} \equiv 0.18$ for reduced $\chi^2$ distribution 
with 60 degrees of freedom. A value of <$\chi^2/\nu$> = 2.1, which is 5$\sigma$ \sout{away} from the 
band of isotropic emission is set as the pattern distinction criteria, and the number of events 
$N^{/}$ which is required to achieve that <$\chi^2/\nu$> value is calculated\RanComment{I dont understand this last sentence}. The values of 
$N^{/}$ \RanComment{What is $N^{/}$} for all the emission patterns \sout{which are listed in Table~\ref{tab:AnisoPattern}}\RanComment{(see Table~\ref{tab:AnisoPattern})} are shown 
in the left panel of Fig.~\ref{fig:convergence}. The values of $N^{/}$, when the fraction of anisotropic 
emission is varied \sout{in}\RanComment{for} pattern 1 and pattern 9, are shown in the right panel of Fig.~\ref{fig:convergence}. 

A simulation with two typical sources that emit uniformly in 4$\pi$, a 10 $\mu$Ci $^{137}Cs$ $\gamma$ 
source (662 keV), and a 2.7 $\mu$Ci AmBe neutron source (5 MeV), shows that for an yield of 
50photons/event it would take about 16 days for electron recoil (ER) events and about 0.8 days for 
nuclear recoil (NR) events to reach \RanComment{a 5$\sigma$ effect} \sout{$10^4$ events}. Therefore a system that can operate stably for few weeks 
is expected to do reasonable measurements for ER and NR events.\RanComment{
I think the important property is the sensitivty and not the number of events as I changed it, also I would reverse the order, stating that in order to get this number of events from typical sources such as .. It would require ~16 days ...}

%%%%%%%%%%%%%%%%%%%%%%%%%%%%%%%%%%%%%%%%%%%%%%%%%%%%%%%%%%%%%%%%%%%%%%%%%
\begin{figure}[h]
\centerline{\includegraphics[width=0.5\linewidth]{Pattern1.pdf}}
\caption{<$\chi^2/\nu$> and its $2\sigma$ band for isotropic emission (black) and for pattern 1 (red). 
The number of events needed to discriminate the pattern is defined as that for which 
the $2\sigma$ band crosses <$\chi^2/\nu$> = 2.1, i.e., $5\sigma$ away from the <$\chi^2/\nu$> for isotropy. }
\label{fig:pattern1}
\end{figure}
%%%%%%%%%%%%%%%%%%%%%%%%%%%%%%%%%%%%%%%%%%%%%%%%%%%%%%%%%%%%%%%%%%%%%%%%%%%%

%%%%%%%%%%%%%%%%%%%%%%%%%%%%%%%%%%%%%%%%%%%%%%%%%%%%%%%%%%%%%%%%%%%%%%%%%
\begin{figure}[h]
\centering
\includegraphics[width=0.45\linewidth]{ConvergenceVsPattern.pdf}
\hspace{0.1cm}
\includegraphics[width=0.45\linewidth]{ConvergenceVsRaniso.pdf}
\caption{(Left) The number of events needed for each emission pattern to cross <$\chi^2/\nu$> = 2.1. Here 
$r_{aniso}$ = 0.1. (Right) The number of events needed for pattern 1 and pattern 9 to cross <$\chi^2/\nu$> = 2.1 
for four different values of $r_{aniso}$.}
\label{fig:convergence}
\end{figure}
%%%%%%%%%%%%%%%%%%%%%%%%%%%%%%%%%%%%%%%%%%%%%%%%%%%%%%%%%%%%%%%%%%%%%%%%%%%%


