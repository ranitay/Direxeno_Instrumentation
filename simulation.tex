\section{Simulation (\mmd{in progress})}
\label{sec:sim}
%%%%%%%%%%%%%%%%%%%%%%%%%%%%%%%%%%%%%%%%%%%%%%%%%%%%%%%%%%%%%%%%%%%%%%%%%%%%%%%%%%%%%%%%%%%%%%%%%%%%%%%%%%%%%%%%%%%%%%%%
A GEANT4 based simulation of the detector is caried out which uses realistic optical parameters, 
and takes into account the physics processes of the propagation of the photons produced in the LXe 
and propagating to the PMTs via the HPFS sphere and vacuum. Assuming various pattern of the LXe 
scintillation, the photons detected at the PMTs are mapped and put through a statistical test to 
check the detector's sensitivity towards those patterns.

The realistic dimension of the detector assembly has been used in the simulation. The important 
geometrical and optical parameters used in the simulation are listed in Table~\ref{tab:OptPar}. 
In the simulation 
framework, The probability for a photon being transmitted/reflected at a given surface is 
determined by Fresnel's equations, which include Snell's law for the transmitted light, 
and specular reflection for the reflected light. The boundary surfaces between different media
such as, the LXe--HPFS, HPFS--vacuum and vacuum--PMT, are considered to be perfectly smooth, 
therefore enabling only specular reflection.

%%%%%%%%%%%%%%%%%%%%%%%%%%%%%%%%%%%%%%%%%%%%%%%%%%%%%%%%%%%%%%%%%%%%%%%%%%%%%%%%%%%%%%%
\begin{table}[h]
  \centering
  \caption{The parameters used in simulation}
  \label{tab:OptPar}
  \begin{tabular}{|c c||c c|}
  \hline
  Parameter & Value & Parameter & Value \\
  \hline
  LXe absorption length & 100 cm & No. of PMT & 20\\
  %\hline
  LXe scattering length & 35 cm & PMT active area & 22mm $\times$ 22mm\\
  %\hline
  HPFS absorption length & 100 cm & PMT QE & $\geq$ 30\% \\
 % \hline
  HPFS scattering length & $\infty$ & PMT distance from centre & 39 mm\\
 % \hline
  LXe refractive index & 1.61 & LXe bubble radius & 1cm\\
  %\hline
  HPFS refractive index & 1.57 & HPFS shell thickness & 2 cm \\
  Scintillation wavelength & 178 $\pm$ 14 nm & Invar tube diameter & 1 mm\\
  \hline
 \end{tabular}
\end{table}
%%%%%%%%%%%%%%%%%%%%%%%%%%%%%%%%%%%%%%%%%%%%%%%%%%%%%%%%%%%%%%%%%%%%%%%%%%%%%%%%%%%%%%%

The photons reaching the PMTs can either be detected or absorbed or reflected from the photocathode 
or the PMT window. In the simulation, a simplified approach of the above possibilities is considered -- 
A photon on the PMT has 30\% probability to be detected (since QE $\geq$ 30\%), 50\% probability to get 
absorbed and 20\% probability to get specularly reflected. As the scintillation light in a particular 
event is emitted by a cloud of excited diatomic molecules, with a linear size much smaller than that 
of the optical system, we simulate each event as a number of photons that are emitted from a point 
in the LXe. The events are uniformly generated in the LXe volume to simulate an entirely radiated 
target much smaller than the mean free path length of the source particles in LXe.


