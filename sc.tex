
We use a time-series server based on influxdb (cite: https://www.influxdata.com/). This time series database is built specifically for handling metrics and events or measurements that are time-stamped. 
For monitoring and visualization we use  
Grafana (cite: https://grafana.com), an open source software for time series analytics. 

The monitored data is collected from a variety of sensors and streamed to the database using the influxdb API integrated in python control scripts.
It includes cryogenic system parameters: several temperatures  measured using a PT100 and a cryoconXXXX taken in different points around the sphere 
and in the pools. To gain better understanding of the LXe we increased the number of temperature monitored positions by using a 2-wired 
readout for some sensors, thus allowing the readout of more PT100 sensors with the same number of feedthroughs data channels.
The temperature-resistance calibration curve were shifted accordingly.   
Pressure/Vacuum reading of the outer- and inner-chamber were done using xxxx and xxxx sensors. 
Termistors based sensors were installed on the cooling water lines, the compressor and around the laboratory for monitoring of the cryocooler operation. They were read using an arduino. 

The xenon flow rate was measured using a xxx. A specially designed switch enabeled the transition between Filling/Circulation/Recooperation modes, thus 
allowing the proper handling of the flow rate data in the data base (either adding to the total Xe amount, removing).


The PMTs voltage, current and trigger rate is also monitoted and streamed to the database. 

An optical camera installed. diff ertc.


